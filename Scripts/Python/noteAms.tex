




% Default to the notebook output style


% Inherit from the specified cell style.






    
\documentclass{article}

    
        
    
    \usepackage{graphicx} % Used to insert images
    \usepackage{adjustbox} % Used to constrain images to a maximum size 
    \usepackage{color} % Allow colors to be defined
    \usepackage{enumerate} % Needed for markdown enumerations to work
    \usepackage{geometry} % Used to adjust the document margins
    \usepackage{amsmath} % Equations
    \usepackage{amssymb} % Equations
    \usepackage{eurosym} % defines \euro
    \usepackage[mathletters]{ucs} % Extended unicode (utf-8) support
    \usepackage[utf8x]{inputenc} % Allow utf-8 characters in the tex document
    \usepackage{fancyvrb} % verbatim replacement that allows latex
    \usepackage{grffile} % extends the file name processing of package graphics 
                         % to support a larger range 
    % The hyperref package gives us a pdf with properly built
    % internal navigation ('pdf bookmarks' for the table of contents,
    % internal cross-reference links, web links for URLs, etc.)
    \usepackage{hyperref}
    \usepackage{longtable} % longtable support required by pandoc >1.10
    \usepackage{booktabs}  % table support for pandoc > 1.12.2
    

    \usepackage{tikz} % Needed to box output/input
    \usepackage{scrextend} % Used to indent output
    \usepackage{needspace} % Make prompts follow contents
    \usepackage{framed} % Used to draw output that spans multiple pages


    
    
    \definecolor{orange}{cmyk}{0,0.4,0.8,0.2}
    \definecolor{darkorange}{rgb}{.71,0.21,0.01}
    \definecolor{darkgreen}{rgb}{.12,.54,.11}
    \definecolor{myteal}{rgb}{.26, .44, .56}
    \definecolor{gray}{gray}{0.45}
    \definecolor{lightgray}{gray}{.95}
    \definecolor{mediumgray}{gray}{.8}
    \definecolor{inputbackground}{rgb}{.95, .95, .85}
    \definecolor{outputbackground}{rgb}{.95, .95, .95}
    \definecolor{traceback}{rgb}{1, .95, .95}
    % ansi colors
    \definecolor{red}{rgb}{.6,0,0}
    \definecolor{green}{rgb}{0,.65,0}
    \definecolor{brown}{rgb}{0.6,0.6,0}
    \definecolor{blue}{rgb}{0,.145,.698}
    \definecolor{purple}{rgb}{.698,.145,.698}
    \definecolor{cyan}{rgb}{0,.698,.698}
    \definecolor{lightgray}{gray}{0.5}
    
    % bright ansi colors
    \definecolor{darkgray}{gray}{0.25}
    \definecolor{lightred}{rgb}{1.0,0.39,0.28}
    \definecolor{lightgreen}{rgb}{0.48,0.99,0.0}
    \definecolor{lightblue}{rgb}{0.53,0.81,0.92}
    \definecolor{lightpurple}{rgb}{0.87,0.63,0.87}
    \definecolor{lightcyan}{rgb}{0.5,1.0,0.83}
    
    % commands and environments needed by pandoc snippets
    % extracted from the output of `pandoc -s`
    \providecommand{\tightlist}{%
      \setlength{\itemsep}{0pt}\setlength{\parskip}{0pt}}
    \DefineVerbatimEnvironment{Highlighting}{Verbatim}{commandchars=\\\{\}}
    % Add ',fontsize=\small' for more characters per line
    \newenvironment{Shaded}{}{}
    \newcommand{\KeywordTok}[1]{\textcolor[rgb]{0.00,0.44,0.13}{\textbf{{#1}}}}
    \newcommand{\DataTypeTok}[1]{\textcolor[rgb]{0.56,0.13,0.00}{{#1}}}
    \newcommand{\DecValTok}[1]{\textcolor[rgb]{0.25,0.63,0.44}{{#1}}}
    \newcommand{\BaseNTok}[1]{\textcolor[rgb]{0.25,0.63,0.44}{{#1}}}
    \newcommand{\FloatTok}[1]{\textcolor[rgb]{0.25,0.63,0.44}{{#1}}}
    \newcommand{\CharTok}[1]{\textcolor[rgb]{0.25,0.44,0.63}{{#1}}}
    \newcommand{\StringTok}[1]{\textcolor[rgb]{0.25,0.44,0.63}{{#1}}}
    \newcommand{\CommentTok}[1]{\textcolor[rgb]{0.38,0.63,0.69}{\textit{{#1}}}}
    \newcommand{\OtherTok}[1]{\textcolor[rgb]{0.00,0.44,0.13}{{#1}}}
    \newcommand{\AlertTok}[1]{\textcolor[rgb]{1.00,0.00,0.00}{\textbf{{#1}}}}
    \newcommand{\FunctionTok}[1]{\textcolor[rgb]{0.02,0.16,0.49}{{#1}}}
    \newcommand{\RegionMarkerTok}[1]{{#1}}
    \newcommand{\ErrorTok}[1]{\textcolor[rgb]{1.00,0.00,0.00}{\textbf{{#1}}}}
    \newcommand{\NormalTok}[1]{{#1}}
    
    % Define a nice break command that doesn't care if a line doesn't already
    % exist.
    \def\br{\hspace*{\fill} \\* }
    % Math Jax compatability definitions
    \def\gt{>}
    \def\lt{<}
    % Document parameters
    \title{noteAms}
    
    
    

    % Pygments definitions
    
\makeatletter
\def\PY@reset{\let\PY@it=\relax \let\PY@bf=\relax%
    \let\PY@ul=\relax \let\PY@tc=\relax%
    \let\PY@bc=\relax \let\PY@ff=\relax}
\def\PY@tok#1{\csname PY@tok@#1\endcsname}
\def\PY@toks#1+{\ifx\relax#1\empty\else%
    \PY@tok{#1}\expandafter\PY@toks\fi}
\def\PY@do#1{\PY@bc{\PY@tc{\PY@ul{%
    \PY@it{\PY@bf{\PY@ff{#1}}}}}}}
\def\PY#1#2{\PY@reset\PY@toks#1+\relax+\PY@do{#2}}

\expandafter\def\csname PY@tok@gt\endcsname{\def\PY@tc##1{\textcolor[rgb]{0.00,0.27,0.87}{##1}}}
\expandafter\def\csname PY@tok@w\endcsname{\def\PY@tc##1{\textcolor[rgb]{0.73,0.73,0.73}{##1}}}
\expandafter\def\csname PY@tok@mb\endcsname{\def\PY@tc##1{\textcolor[rgb]{0.40,0.40,0.40}{##1}}}
\expandafter\def\csname PY@tok@sh\endcsname{\def\PY@tc##1{\textcolor[rgb]{0.73,0.13,0.13}{##1}}}
\expandafter\def\csname PY@tok@go\endcsname{\def\PY@tc##1{\textcolor[rgb]{0.53,0.53,0.53}{##1}}}
\expandafter\def\csname PY@tok@mh\endcsname{\def\PY@tc##1{\textcolor[rgb]{0.40,0.40,0.40}{##1}}}
\expandafter\def\csname PY@tok@ow\endcsname{\let\PY@bf=\textbf\def\PY@tc##1{\textcolor[rgb]{0.67,0.13,1.00}{##1}}}
\expandafter\def\csname PY@tok@sx\endcsname{\def\PY@tc##1{\textcolor[rgb]{0.00,0.50,0.00}{##1}}}
\expandafter\def\csname PY@tok@mi\endcsname{\def\PY@tc##1{\textcolor[rgb]{0.40,0.40,0.40}{##1}}}
\expandafter\def\csname PY@tok@c\endcsname{\let\PY@it=\textit\def\PY@tc##1{\textcolor[rgb]{0.25,0.50,0.50}{##1}}}
\expandafter\def\csname PY@tok@s\endcsname{\def\PY@tc##1{\textcolor[rgb]{0.73,0.13,0.13}{##1}}}
\expandafter\def\csname PY@tok@nv\endcsname{\def\PY@tc##1{\textcolor[rgb]{0.10,0.09,0.49}{##1}}}
\expandafter\def\csname PY@tok@s2\endcsname{\def\PY@tc##1{\textcolor[rgb]{0.73,0.13,0.13}{##1}}}
\expandafter\def\csname PY@tok@kn\endcsname{\let\PY@bf=\textbf\def\PY@tc##1{\textcolor[rgb]{0.00,0.50,0.00}{##1}}}
\expandafter\def\csname PY@tok@sr\endcsname{\def\PY@tc##1{\textcolor[rgb]{0.73,0.40,0.53}{##1}}}
\expandafter\def\csname PY@tok@ss\endcsname{\def\PY@tc##1{\textcolor[rgb]{0.10,0.09,0.49}{##1}}}
\expandafter\def\csname PY@tok@ni\endcsname{\let\PY@bf=\textbf\def\PY@tc##1{\textcolor[rgb]{0.60,0.60,0.60}{##1}}}
\expandafter\def\csname PY@tok@sb\endcsname{\def\PY@tc##1{\textcolor[rgb]{0.73,0.13,0.13}{##1}}}
\expandafter\def\csname PY@tok@kc\endcsname{\let\PY@bf=\textbf\def\PY@tc##1{\textcolor[rgb]{0.00,0.50,0.00}{##1}}}
\expandafter\def\csname PY@tok@s1\endcsname{\def\PY@tc##1{\textcolor[rgb]{0.73,0.13,0.13}{##1}}}
\expandafter\def\csname PY@tok@gd\endcsname{\def\PY@tc##1{\textcolor[rgb]{0.63,0.00,0.00}{##1}}}
\expandafter\def\csname PY@tok@gh\endcsname{\let\PY@bf=\textbf\def\PY@tc##1{\textcolor[rgb]{0.00,0.00,0.50}{##1}}}
\expandafter\def\csname PY@tok@vc\endcsname{\def\PY@tc##1{\textcolor[rgb]{0.10,0.09,0.49}{##1}}}
\expandafter\def\csname PY@tok@gp\endcsname{\let\PY@bf=\textbf\def\PY@tc##1{\textcolor[rgb]{0.00,0.00,0.50}{##1}}}
\expandafter\def\csname PY@tok@c1\endcsname{\let\PY@it=\textit\def\PY@tc##1{\textcolor[rgb]{0.25,0.50,0.50}{##1}}}
\expandafter\def\csname PY@tok@kp\endcsname{\def\PY@tc##1{\textcolor[rgb]{0.00,0.50,0.00}{##1}}}
\expandafter\def\csname PY@tok@sc\endcsname{\def\PY@tc##1{\textcolor[rgb]{0.73,0.13,0.13}{##1}}}
\expandafter\def\csname PY@tok@nt\endcsname{\let\PY@bf=\textbf\def\PY@tc##1{\textcolor[rgb]{0.00,0.50,0.00}{##1}}}
\expandafter\def\csname PY@tok@vg\endcsname{\def\PY@tc##1{\textcolor[rgb]{0.10,0.09,0.49}{##1}}}
\expandafter\def\csname PY@tok@gu\endcsname{\let\PY@bf=\textbf\def\PY@tc##1{\textcolor[rgb]{0.50,0.00,0.50}{##1}}}
\expandafter\def\csname PY@tok@sd\endcsname{\let\PY@it=\textit\def\PY@tc##1{\textcolor[rgb]{0.73,0.13,0.13}{##1}}}
\expandafter\def\csname PY@tok@err\endcsname{\def\PY@bc##1{\setlength{\fboxsep}{0pt}\fcolorbox[rgb]{1.00,0.00,0.00}{1,1,1}{\strut ##1}}}
\expandafter\def\csname PY@tok@o\endcsname{\def\PY@tc##1{\textcolor[rgb]{0.40,0.40,0.40}{##1}}}
\expandafter\def\csname PY@tok@mo\endcsname{\def\PY@tc##1{\textcolor[rgb]{0.40,0.40,0.40}{##1}}}
\expandafter\def\csname PY@tok@gi\endcsname{\def\PY@tc##1{\textcolor[rgb]{0.00,0.63,0.00}{##1}}}
\expandafter\def\csname PY@tok@si\endcsname{\let\PY@bf=\textbf\def\PY@tc##1{\textcolor[rgb]{0.73,0.40,0.53}{##1}}}
\expandafter\def\csname PY@tok@kr\endcsname{\let\PY@bf=\textbf\def\PY@tc##1{\textcolor[rgb]{0.00,0.50,0.00}{##1}}}
\expandafter\def\csname PY@tok@mf\endcsname{\def\PY@tc##1{\textcolor[rgb]{0.40,0.40,0.40}{##1}}}
\expandafter\def\csname PY@tok@nc\endcsname{\let\PY@bf=\textbf\def\PY@tc##1{\textcolor[rgb]{0.00,0.00,1.00}{##1}}}
\expandafter\def\csname PY@tok@na\endcsname{\def\PY@tc##1{\textcolor[rgb]{0.49,0.56,0.16}{##1}}}
\expandafter\def\csname PY@tok@gr\endcsname{\def\PY@tc##1{\textcolor[rgb]{1.00,0.00,0.00}{##1}}}
\expandafter\def\csname PY@tok@ge\endcsname{\let\PY@it=\textit}
\expandafter\def\csname PY@tok@se\endcsname{\let\PY@bf=\textbf\def\PY@tc##1{\textcolor[rgb]{0.73,0.40,0.13}{##1}}}
\expandafter\def\csname PY@tok@nf\endcsname{\def\PY@tc##1{\textcolor[rgb]{0.00,0.00,1.00}{##1}}}
\expandafter\def\csname PY@tok@kt\endcsname{\def\PY@tc##1{\textcolor[rgb]{0.69,0.00,0.25}{##1}}}
\expandafter\def\csname PY@tok@cs\endcsname{\let\PY@it=\textit\def\PY@tc##1{\textcolor[rgb]{0.25,0.50,0.50}{##1}}}
\expandafter\def\csname PY@tok@gs\endcsname{\let\PY@bf=\textbf}
\expandafter\def\csname PY@tok@k\endcsname{\let\PY@bf=\textbf\def\PY@tc##1{\textcolor[rgb]{0.00,0.50,0.00}{##1}}}
\expandafter\def\csname PY@tok@vi\endcsname{\def\PY@tc##1{\textcolor[rgb]{0.10,0.09,0.49}{##1}}}
\expandafter\def\csname PY@tok@kd\endcsname{\let\PY@bf=\textbf\def\PY@tc##1{\textcolor[rgb]{0.00,0.50,0.00}{##1}}}
\expandafter\def\csname PY@tok@nl\endcsname{\def\PY@tc##1{\textcolor[rgb]{0.63,0.63,0.00}{##1}}}
\expandafter\def\csname PY@tok@ne\endcsname{\let\PY@bf=\textbf\def\PY@tc##1{\textcolor[rgb]{0.82,0.25,0.23}{##1}}}
\expandafter\def\csname PY@tok@bp\endcsname{\def\PY@tc##1{\textcolor[rgb]{0.00,0.50,0.00}{##1}}}
\expandafter\def\csname PY@tok@nb\endcsname{\def\PY@tc##1{\textcolor[rgb]{0.00,0.50,0.00}{##1}}}
\expandafter\def\csname PY@tok@il\endcsname{\def\PY@tc##1{\textcolor[rgb]{0.40,0.40,0.40}{##1}}}
\expandafter\def\csname PY@tok@nn\endcsname{\let\PY@bf=\textbf\def\PY@tc##1{\textcolor[rgb]{0.00,0.00,1.00}{##1}}}
\expandafter\def\csname PY@tok@no\endcsname{\def\PY@tc##1{\textcolor[rgb]{0.53,0.00,0.00}{##1}}}
\expandafter\def\csname PY@tok@nd\endcsname{\def\PY@tc##1{\textcolor[rgb]{0.67,0.13,1.00}{##1}}}
\expandafter\def\csname PY@tok@cp\endcsname{\def\PY@tc##1{\textcolor[rgb]{0.74,0.48,0.00}{##1}}}
\expandafter\def\csname PY@tok@cm\endcsname{\let\PY@it=\textit\def\PY@tc##1{\textcolor[rgb]{0.25,0.50,0.50}{##1}}}
\expandafter\def\csname PY@tok@m\endcsname{\def\PY@tc##1{\textcolor[rgb]{0.40,0.40,0.40}{##1}}}

\def\PYZbs{\char`\\}
\def\PYZus{\char`\_}
\def\PYZob{\char`\{}
\def\PYZcb{\char`\}}
\def\PYZca{\char`\^}
\def\PYZam{\char`\&}
\def\PYZlt{\char`\<}
\def\PYZgt{\char`\>}
\def\PYZsh{\char`\#}
\def\PYZpc{\char`\%}
\def\PYZdl{\char`\$}
\def\PYZhy{\char`\-}
\def\PYZsq{\char`\'}
\def\PYZdq{\char`\"}
\def\PYZti{\char`\~}
% for compatibility with earlier versions
\def\PYZat{@}
\def\PYZlb{[}
\def\PYZrb{]}
\makeatother


    % NB prompt colors
    \definecolor{nbframe-border}{rgb}{0.867,0.867,0.867}
    \definecolor{nbframe-bg}{rgb}{0.969,0.969,0.969}
    \definecolor{nbframe-in-prompt}{rgb}{0.0,0.0,0.502}
    \definecolor{nbframe-out-prompt}{rgb}{0.545,0.0,0.0}

    % NB prompt lengths
    \newlength{\inputpadding}
    \setlength{\inputpadding}{0.5em}
    \newlength{\cellleftmargin}
    \setlength{\cellleftmargin}{0.15\linewidth}
    \newlength{\borderthickness}
    \setlength{\borderthickness}{0.4pt}
    \newlength{\smallerfontscale}
    \setlength{\smallerfontscale}{9.5pt}

    % NB prompt font size
    \def\smaller{\fontsize{\smallerfontscale}{\smallerfontscale}\selectfont}

    % Define a background layer, in which the nb prompt shape is drawn
    \pgfdeclarelayer{background}
    \pgfsetlayers{background,main}
    \usetikzlibrary{calc}

    % define styles for the normal border and the torn border
    \tikzset{
      normal border/.style={draw=nbframe-border, fill=nbframe-bg,
        rectangle, rounded corners=2.5pt, line width=\borderthickness},
      torn border/.style={draw=white, fill=white, line width=\borderthickness}}

    % Macro to draw the shape behind the text, when it fits completly in the
    % page
    \def\notebookcellframe#1{%
    \tikz{%
      \node[inner sep=\inputpadding] (A) {#1};% Draw the text of the node
      \begin{pgfonlayer}{background}% Draw the shape behind
      \fill[normal border]%
            (A.south east) -- ($(A.south west)+(\cellleftmargin,0)$) -- 
            ($(A.north west)+(\cellleftmargin,0)$) -- (A.north east) -- cycle;
      \end{pgfonlayer}}}%

    % Macro to draw the shape, when the text will continue in next page
    \def\notebookcellframetop#1{%
    \tikz{%
      \node[inner sep=\inputpadding] (A) {#1};    % Draw the text of the node
      \begin{pgfonlayer}{background}    
      \fill[normal border]              % Draw the ``complete shape'' behind
            (A.south east) -- ($(A.south west)+(\cellleftmargin,0)$) -- 
            ($(A.north west)+(\cellleftmargin,0)$) -- (A.north east) -- cycle;
      \fill[torn border]                % Add the torn lower border
            ($(A.south east)-(0,.1)$) -- ($(A.south west)+(\cellleftmargin,-.1)$) -- 
            ($(A.south west)+(\cellleftmargin,.1)$) -- ($(A.south east)+(0,.1)$) -- cycle;
      \end{pgfonlayer}}}

    % Macro to draw the shape, when the text continues from previous page
    \def\notebookcellframebottom#1{%
    \tikz{%
      \node[inner sep=\inputpadding] (A) {#1};   % Draw the text of the node
      \begin{pgfonlayer}{background}   
      \fill[normal border]             % Draw the ``complete shape'' behind
            (A.south east) -- ($(A.south west)+(\cellleftmargin,0)$) -- 
            ($(A.north west)+(\cellleftmargin,0)$) -- (A.north east) -- cycle;
      \fill[torn border]               % Add the torn upper border
            ($(A.north east)-(0,.1)$) -- ($(A.north west)+(\cellleftmargin,-.1)$) -- 
            ($(A.north west)+(\cellleftmargin,.1)$) -- ($(A.north east)+(0,.1)$) -- cycle;
      \end{pgfonlayer}}}

    % Macro to draw the shape, when both the text continues from previous page
    % and it will continue in next page
    \def\notebookcellframemiddle#1{%
    \tikz{%
      \node[inner sep=\inputpadding] (A) {#1};   % Draw the text of the node
      \begin{pgfonlayer}{background}   
      \fill[normal border]             % Draw the ``complete shape'' behind
            (A.south east) -- ($(A.south west)+(\cellleftmargin,0)$) -- 
            ($(A.north west)+(\cellleftmargin,0)$) -- (A.north east) -- cycle;
      \fill[torn border]               % Add the torn lower border
            ($(A.south east)-(0,.1)$) -- ($(A.south west)+(\cellleftmargin,-.1)$) -- 
            ($(A.south west)+(\cellleftmargin,.1)$) -- ($(A.south east)+(0,.1)$) -- cycle;
      \fill[torn border]               % Add the torn upper border
            ($(A.north east)-(0,.1)$) -- ($(A.north west)+(\cellleftmargin,-.1)$) -- 
            ($(A.north west)+(\cellleftmargin,.1)$) -- ($(A.north east)+(0,.1)$) -- cycle;
      \end{pgfonlayer}}}

    % Define the environment which puts the frame
    % In this case, the environment also accepts an argument with an optional
    % title (which defaults to ``Example'', which is typeset in a box overlaid
    % on the top border
    \newenvironment{notebookcell}[1][0]{%
      \def\FrameCommand{\notebookcellframe}%
      \def\FirstFrameCommand{\notebookcellframetop}%
      \def\LastFrameCommand{\notebookcellframebottom}%
      \def\MidFrameCommand{\notebookcellframemiddle}%
      \par\vspace{1\baselineskip}%
      \MakeFramed {\FrameRestore}%
      \noindent\tikz\node[inner sep=0em] at ($(A.north west)-(0,0)$) {%
      
      }; 
      \par}%
    {\endMakeFramed}



    
    % Prevent overflowing lines due to hard-to-break entities
    \sloppy 
    % Setup hyperref package
    \hypersetup{
      breaklinks=true,  % so long urls are correctly broken across lines
      colorlinks=true,
      urlcolor=blue,
      linkcolor=darkorange,
      citecolor=darkgreen,
      }
    % Slightly bigger margins than the latex defaults
    
    \geometry{verbose,tmargin=1in,bmargin=1in,lmargin=1in,rmargin=1in}
    
    

    \begin{document}
    
    
    \maketitle
    
    

    %Date: 04.01.2016
    % Add contents below.

{\par%
\vspace{-1\baselineskip}%
\needspace{4\baselineskip}}%
\begin{notebookcell}[]%
\begin{addmargin}[\cellleftmargin]{0em}% left, right
{\smaller%
\par%
%
\vspace{-1\smallerfontscale}%
\begin{Verbatim}[commandchars=\\\{\}]
\PY{k+kn}{import} \PY{n+nn}{numpy} \PY{k}{as} \PY{n+nn}{np}
\PY{k+kn}{import} \PY{n+nn}{matplotlib}\PY{n+nn}{.}\PY{n+nn}{pyplot} \PY{k}{as} \PY{n+nn}{plt}
\PY{k+kn}{import} \PY{n+nn}{math}
\end{Verbatim}
%
\par%
\vspace{-1\smallerfontscale}}%
\end{addmargin}
\end{notebookcell}


    Approximate Median Significance (AMS) defined as:
\(AMS = \sqrt{2 { (s + b + b_r) log[1 + (s/(b+b_{reg}))] - s}}\)

where

\begin{itemize}
\tightlist
\item
  \(b_{reg} = 10\) is a regulization term (set by the contest),
\item
  \(b = \sum_{i=1}^{n} w_i, y_i=0\) is sum of weighted background
  (incorrectly classified as signal),
\item
  \(s = \sum_{i=1}^{n} w_i, y_i=1\) is sum of weighted signals
  (correctly classified as signal),
\item
  \(log\) is natural logarithm
\end{itemize}

    % Add contents below.

{\par%
\vspace{-1\baselineskip}%
\needspace{4\baselineskip}}%
\begin{notebookcell}[]%
\begin{addmargin}[\cellleftmargin]{0em}% left, right
{\smaller%
\par%
%
\vspace{-1\smallerfontscale}%
\begin{Verbatim}[commandchars=\\\{\}]
\PY{k}{def} \PY{n+nf}{calcAMS}\PY{p}{(}\PY{n}{s}\PY{p}{,}\PY{n}{b}\PY{p}{)}\PY{p}{:}    
    \PY{n}{br} \PY{o}{=} \PY{l+m+mf}{10.0}
    \PY{n}{radicand} \PY{o}{=} \PY{l+m+mi}{2} \PY{o}{*}\PY{p}{(} \PY{p}{(}\PY{n}{s}\PY{o}{+}\PY{n}{b}\PY{o}{+}\PY{n}{br}\PY{p}{)} \PY{o}{*} \PY{n}{math}\PY{o}{.}\PY{n}{log} \PY{p}{(}\PY{l+m+mf}{1.0} \PY{o}{+} \PY{n}{s}\PY{o}{/}\PY{p}{(}\PY{n}{b}\PY{o}{+}\PY{n}{br}\PY{p}{)}\PY{p}{)} \PY{o}{\PYZhy{}}\PY{n}{s}\PY{p}{)}
    \PY{k}{if} \PY{n}{radicand} \PY{o}{\PYZlt{}} \PY{l+m+mi}{0}\PY{p}{:}
        \PY{n+nb}{print}\PY{p}{(}\PY{l+s}{\PYZsq{}}\PY{l+s}{radicand is negative. Exiting}\PY{l+s}{\PYZsq{}}\PY{p}{)}
        \PY{n}{exit}\PY{p}{(}\PY{p}{)}
    \PY{k}{else}\PY{p}{:}
        \PY{n}{ams} \PY{o}{=} \PY{n}{math}\PY{o}{.}\PY{n}{sqrt}\PY{p}{(}\PY{n}{radicand}\PY{p}{)}
        \PY{n+nb}{print}\PY{p}{(}\PY{l+s}{\PYZdq{}}\PY{l+s}{AMS:}\PY{l+s}{\PYZdq{}}\PY{p}{,} \PY{n}{ams}\PY{p}{)}
        \PY{k}{return} \PY{n}{ams}
\end{Verbatim}
%
\par%
\vspace{-1\smallerfontscale}}%
\end{addmargin}
\end{notebookcell}


    Following this definition, we can derive a maximum AMS by simply summing
the weights of all positive labels.

    % Add contents below.

{\par%
\vspace{-1\baselineskip}%
\needspace{4\baselineskip}}%
\begin{notebookcell}[]%
\begin{addmargin}[\cellleftmargin]{0em}% left, right
{\smaller%
\par%
%
\vspace{-1\smallerfontscale}%
\begin{Verbatim}[commandchars=\\\{\}]
\PY{k}{def} \PY{n+nf}{calcWeightSums}\PY{p}{(}\PY{n}{weights}\PY{p}{,}\PY{n}{preds}\PY{p}{,}\PY{n}{labels}\PY{p}{)}\PY{p}{:}
    \PY{n}{s} \PY{o}{=} \PY{l+m+mi}{0}
    \PY{n}{b} \PY{o}{=} \PY{l+m+mi}{0}
    \PY{k}{for} \PY{n}{j} \PY{o+ow}{in} \PY{n+nb}{list}\PY{p}{(}\PY{n+nb}{range}\PY{p}{(}\PY{l+m+mi}{0}\PY{p}{,}\PY{n+nb}{len}\PY{p}{(}\PY{n}{preds}\PY{p}{)}\PY{p}{)}\PY{p}{)}\PY{p}{:}
        \PY{n}{pred} \PY{o}{=} \PY{n}{preds}\PY{p}{[}\PY{n}{j}\PY{p}{]}
        \PY{n}{label} \PY{o}{=} \PY{n}{labels}\PY{p}{[}\PY{n}{j}\PY{p}{]}
        \PY{n}{weight} \PY{o}{=} \PY{n}{weights}\PY{p}{[}\PY{n}{j}\PY{p}{]}
        \PY{k}{if} \PY{n}{pred} \PY{o}{\PYZgt{}} \PY{l+m+mf}{0.}\PY{p}{:}
            \PY{k}{if} \PY{n}{label} \PY{o}{\PYZgt{}} \PY{l+m+mf}{0.}\PY{p}{:}
                \PY{n}{s} \PY{o}{+}\PY{o}{=} \PY{n}{weight}
            \PY{k}{else}\PY{p}{:}
                \PY{n}{b} \PY{o}{+}\PY{o}{=} \PY{n}{weight}
    \PY{k}{return} \PY{n}{s}\PY{p}{,}\PY{n}{b}
\end{Verbatim}
%
\par%
\vspace{-1\smallerfontscale}}%
\end{addmargin}
\end{notebookcell}


    % Add contents below.

{\par%
\vspace{-1\baselineskip}%
\needspace{4\baselineskip}}%
\begin{notebookcell}[]%
\begin{addmargin}[\cellleftmargin]{0em}% left, right
{\smaller%
\par%
%
\vspace{-1\smallerfontscale}%
\begin{Verbatim}[commandchars=\\\{\}]
\PY{k}{def} \PY{n+nf}{calcMaxAMS}\PY{p}{(}\PY{n}{weights}\PY{p}{,}\PY{n}{labels}\PY{p}{)}\PY{p}{:}
    \PY{n}{s}\PY{p}{,}\PY{n}{b} \PY{o}{=} \PY{n}{calcWeightSums}\PY{p}{(}\PY{n}{weights}\PY{p}{,}\PY{n}{labels}\PY{p}{,}\PY{n}{labels}\PY{p}{)}
    \PY{n}{ams} \PY{o}{=} \PY{n}{calcAMS}\PY{p}{(}\PY{n}{s}\PY{p}{,}\PY{n}{b}\PY{p}{)}
    \PY{n+nb}{print}\PY{p}{(}\PY{l+s}{\PYZdq{}}\PY{l+s}{Found}\PY{l+s}{\PYZdq{}}\PY{p}{,} \PY{n+nb}{int}\PY{p}{(}\PY{n}{labels}\PY{o}{.}\PY{n}{cumsum}\PY{p}{(}\PY{p}{)}\PY{p}{[}\PY{o}{\PYZhy{}}\PY{l+m+mi}{1}\PY{p}{]}\PY{p}{)}\PY{p}{,} \PY{l+s}{\PYZdq{}}\PY{l+s}{signals.}\PY{l+s}{\PYZdq{}}\PY{p}{)}
    \PY{n+nb}{print}\PY{p}{(}\PY{l+s}{\PYZdq{}}\PY{l+s}{Weightsums signal:}\PY{l+s}{\PYZdq{}}\PY{p}{,} \PY{n}{s}\PY{p}{,} \PY{l+s}{\PYZdq{}}\PY{l+s}{| background:}\PY{l+s}{\PYZdq{}}\PY{p}{,} \PY{n}{b}\PY{p}{)}
    \PY{n+nb}{print}\PY{p}{(}\PY{l+s}{\PYZdq{}}\PY{l+s}{Maximum AMS possible with this Data:}\PY{l+s}{\PYZdq{}}\PY{p}{,} \PY{n}{ams}\PY{p}{)}
    \PY{k}{return} \PY{n}{ams}
\end{Verbatim}
%
\par%
\vspace{-1\smallerfontscale}}%
\end{addmargin}
\end{notebookcell}


    We generate AMS with good seperable toy-data, starting with the maximum
AMS. The data of the actual challenge is weighted to punish
wrong-identified signals significantly harder than wrong background. Our
toy-data will do so by using its signal-probability as weight, the
features are randomized by normal distributions.

    % Add contents below.

{\par%
\vspace{-1\baselineskip}%
\needspace{4\baselineskip}}%
\begin{notebookcell}[]%
\begin{addmargin}[\cellleftmargin]{0em}% left, right
{\smaller%
\par%
%
\vspace{-1\smallerfontscale}%
\begin{Verbatim}[commandchars=\\\{\}]
\PY{k}{def} \PY{n+nf}{generateFeature}\PY{p}{(}\PY{n}{label}\PY{p}{,} \PY{n}{mu\PYZus{}s}\PY{p}{,} \PY{n}{mu\PYZus{}b}\PY{p}{,} \PY{n}{sigma\PYZus{}s}\PY{o}{=}\PY{l+m+mi}{5}\PY{p}{,} \PY{n}{sigma\PYZus{}b}\PY{o}{=}\PY{l+m+mi}{5}\PY{p}{)}\PY{p}{:}
    \PY{k}{if} \PY{n}{label} \PY{o+ow}{is} \PY{l+m+mi}{1}\PY{p}{:}
        \PY{n}{mu} \PY{o}{=} \PY{n}{mu\PYZus{}s}
        \PY{n}{sigma} \PY{o}{=} \PY{n}{sigma\PYZus{}s}
    \PY{k}{else}\PY{p}{:}
        \PY{n}{mu} \PY{o}{=} \PY{n}{mu\PYZus{}b}
        \PY{n}{sigma} \PY{o}{=} \PY{n}{sigma\PYZus{}b}
    \PY{k}{return} \PY{n}{np}\PY{o}{.}\PY{n}{random}\PY{o}{.}\PY{n}{normal}\PY{p}{(}\PY{n}{mu}\PY{p}{,}\PY{n}{sigma}\PY{p}{)}
\end{Verbatim}
%
\par%
\vspace{-1\smallerfontscale}}%
\end{addmargin}
\end{notebookcell}


    % Add contents below.

{\par%
\vspace{-1\baselineskip}%
\needspace{4\baselineskip}}%
\begin{notebookcell}[]%
\begin{addmargin}[\cellleftmargin]{0em}% left, right
{\smaller%
\par%
%
\vspace{-1\smallerfontscale}%
\begin{Verbatim}[commandchars=\\\{\}]
\PY{k}{def} \PY{n+nf}{createToyData}\PY{p}{(}\PY{n}{n} \PY{o}{=} \PY{l+m+mi}{100}\PY{p}{,}\PY{n}{dim} \PY{o}{=} \PY{l+m+mi}{3}\PY{p}{,}\PY{n}{s\PYZus{}prob} \PY{o}{=} \PY{l+m+mf}{0.05}\PY{p}{)}\PY{p}{:}
    \PY{n}{data}\PY{o}{=} \PY{n}{np}\PY{o}{.}\PY{n}{zeros}\PY{p}{(}\PY{n}{shape} \PY{o}{=} \PY{p}{(}\PY{n}{n}\PY{p}{,}\PY{n}{dim}\PY{p}{)}\PY{p}{,}\PY{n}{dtype}\PY{o}{=}\PY{n+nb}{float}\PY{p}{)}
    \PY{k}{if} \PY{n}{dim} \PY{o}{\PYZlt{}} \PY{l+m+mi}{3}\PY{p}{:}
        \PY{n+nb}{print}\PY{p}{(}\PY{l+s}{\PYZdq{}}\PY{l+s}{Operation canceled.}\PY{l+s}{\PYZdq{}}\PY{p}{,}
              \PY{l+s}{\PYZdq{}}\PY{l+s}{Data should have at least one}\PY{l+s}{\PYZdq{}}\PY{p}{,}
              \PY{l+s}{\PYZdq{}}\PY{l+s}{additional dimension besides weights and labels.}\PY{l+s}{\PYZdq{}}\PY{p}{,}
              \PY{l+s}{\PYZdq{}}\PY{l+s}{(dim \PYZgt{}=3)}\PY{l+s}{\PYZdq{}}\PY{p}{)}
        \PY{k}{return} \PY{k}{None}
    \PY{n}{data}\PY{p}{[}\PY{p}{:}\PY{p}{,}\PY{l+m+mi}{0}\PY{p}{]} \PY{o}{=} \PY{n}{np}\PY{o}{.}\PY{n}{random}\PY{o}{.}\PY{n}{rand}\PY{p}{(}\PY{n}{n}\PY{p}{)} \PY{c}{\PYZsh{}weights}
    \PY{k}{for} \PY{n}{i} \PY{o+ow}{in} \PY{n+nb}{range}\PY{p}{(}\PY{l+m+mi}{0}\PY{p}{,}\PY{n}{n}\PY{p}{)}\PY{p}{:}
        \PY{k}{if} \PY{n}{data}\PY{p}{[}\PY{n}{i}\PY{p}{,}\PY{l+m+mi}{0}\PY{p}{]} \PY{o}{\PYZlt{}}\PY{o}{=} \PY{n}{s\PYZus{}prob}\PY{p}{:} \PY{c}{\PYZsh{} label\PYZhy{}determination}
            \PY{n}{label} \PY{o}{=} \PY{l+m+mi}{1}
        \PY{k}{else}\PY{p}{:}
            \PY{n}{label} \PY{o}{=} \PY{l+m+mi}{0}
        \PY{n}{data}\PY{p}{[}\PY{n}{i}\PY{p}{,}\PY{l+m+mi}{1}\PY{p}{]} \PY{o}{=} \PY{n}{label}
        \PY{k}{for} \PY{n}{j} \PY{o+ow}{in} \PY{n+nb}{range}\PY{p}{(}\PY{l+m+mi}{2}\PY{p}{,}\PY{n}{dim}\PY{p}{)}\PY{p}{:}
            \PY{c}{\PYZsh{}mu\PYZus{}s=j*5}
            \PY{c}{\PYZsh{}mu\PYZus{}b=j*20}
            \PY{n}{data}\PY{p}{[}\PY{n}{i}\PY{p}{,}\PY{n}{j}\PY{p}{]}\PY{o}{=}\PY{n}{generateFeature}\PY{p}{(}\PY{n}{label}\PY{p}{,}\PY{n}{mu\PYZus{}s}\PY{o}{=}\PY{p}{(}\PY{n}{j}\PY{o}{\PYZhy{}}\PY{l+m+mi}{1}\PY{p}{)}\PY{o}{*}\PY{l+m+mi}{5}\PY{p}{,}\PY{n}{mu\PYZus{}b}\PY{o}{=}\PY{p}{(}\PY{n}{j}\PY{o}{\PYZhy{}}\PY{l+m+mi}{1}\PY{p}{)}\PY{o}{*}\PY{l+m+mi}{20}\PY{p}{)}
    \PY{k}{return} \PY{n}{data}
\end{Verbatim}
%
\par%
\vspace{-1\smallerfontscale}}%
\end{addmargin}
\end{notebookcell}


    % Add contents below.

{\par%
\vspace{-1\baselineskip}%
\needspace{4\baselineskip}}%
\begin{notebookcell}[]%
\begin{addmargin}[\cellleftmargin]{0em}% left, right
{\smaller%
\par%
%
\vspace{-1\smallerfontscale}%
\begin{Verbatim}[commandchars=\\\{\}]
\PY{n}{n} \PY{o}{=} \PY{l+m+mi}{100000}
\PY{n}{prob} \PY{o}{=} \PY{l+m+mf}{0.05}
\PY{n}{data} \PY{o}{=} \PY{n}{createToyData}\PY{p}{(}\PY{n}{n}\PY{p}{,}\PY{n}{dim}\PY{o}{=}\PY{l+m+mi}{10}\PY{p}{,}\PY{n}{s\PYZus{}prob}\PY{o}{=}\PY{n}{prob}\PY{p}{)}
\end{Verbatim}
%
\par%
\vspace{-1\smallerfontscale}}%
\end{addmargin}
\end{notebookcell}


    % Add contents below.

{\par%
\vspace{-1\baselineskip}%
\needspace{4\baselineskip}}%
\begin{notebookcell}[]%
\begin{addmargin}[\cellleftmargin]{0em}% left, right
{\smaller%
\par%
%
\vspace{-1\smallerfontscale}%
\begin{Verbatim}[commandchars=\\\{\}]
\PY{n}{weights} \PY{o}{=} \PY{n}{data}\PY{p}{[}\PY{p}{:}\PY{p}{,}\PY{l+m+mi}{0}\PY{p}{]}
\PY{n}{labels} \PY{o}{=} \PY{n}{data}\PY{p}{[}\PY{p}{:}\PY{p}{,}\PY{l+m+mi}{1}\PY{p}{]}
\PY{n}{calcMaxAMS}\PY{p}{(}\PY{n}{weights}\PY{p}{,}\PY{n}{labels}\PY{p}{)}\PY{p}{;}
\end{Verbatim}
%
\par%
\vspace{-1\smallerfontscale}}%
\end{addmargin}
\end{notebookcell}


    We randomly guess labels for a solution for a second AMS with knowledge
about the toydatas signal-probability.

    % Add contents below.

{\par%
\vspace{-1\baselineskip}%
\needspace{4\baselineskip}}%
\begin{notebookcell}[]%
\begin{addmargin}[\cellleftmargin]{0em}% left, right
{\smaller%
\par%
%
\vspace{-1\smallerfontscale}%
\begin{Verbatim}[commandchars=\\\{\}]
\PY{n}{sol\PYZus{}weights} \PY{o}{=} \PY{n}{np}\PY{o}{.}\PY{n}{random}\PY{o}{.}\PY{n}{rand}\PY{p}{(}\PY{n}{n}\PY{p}{)}
\PY{n}{sol} \PY{o}{=} \PY{n}{np}\PY{o}{.}\PY{n}{zeros}\PY{p}{(}\PY{n}{n}\PY{p}{)}
\PY{k}{for} \PY{n}{i} \PY{o+ow}{in} \PY{n+nb}{range}\PY{p}{(}\PY{l+m+mi}{0}\PY{p}{,}\PY{n}{n}\PY{p}{)}\PY{p}{:}
    \PY{k}{if} \PY{n}{sol\PYZus{}weights}\PY{p}{[}\PY{n}{i}\PY{p}{]} \PY{o}{\PYZlt{}}\PY{o}{=} \PY{n}{prob}\PY{p}{:} \PY{c}{\PYZsh{} label\PYZhy{}determination}
        \PY{n}{sol}\PY{p}{[}\PY{n}{i}\PY{p}{]} \PY{o}{=} \PY{l+m+mi}{1}
    \PY{k}{else}\PY{p}{:}
        \PY{n}{sol}\PY{p}{[}\PY{n}{i}\PY{p}{]} \PY{o}{=} \PY{l+m+mi}{0}
\end{Verbatim}
%
\par%
\vspace{-1\smallerfontscale}}%
\end{addmargin}
\end{notebookcell}


    % Add contents below.

{\par%
\vspace{-1\baselineskip}%
\needspace{4\baselineskip}}%
\begin{notebookcell}[]%
\begin{addmargin}[\cellleftmargin]{0em}% left, right
{\smaller%
\par%
%
\vspace{-1\smallerfontscale}%
\begin{Verbatim}[commandchars=\\\{\}]
\PY{n}{s}\PY{p}{,}\PY{n}{b} \PY{o}{=} \PY{n}{calcWeightSums}\PY{p}{(}\PY{n}{weights}\PY{p}{,}\PY{n}{sol}\PY{p}{,}\PY{n}{labels}\PY{p}{)}
\PY{n}{calcAMS}\PY{p}{(}\PY{n}{s}\PY{p}{,}\PY{n}{b}\PY{p}{)}\PY{p}{;}
\end{Verbatim}
%
\par%
\vspace{-1\smallerfontscale}}%
\end{addmargin}
\end{notebookcell}



    % Add a bibliography block to the postdoc
    
    
    
    \end{document}
