\section{Introduction}\label{ch:intro}\raggedbottom

This chapter first presents the Higgs Boson Machine Learning Challenge and explains its motivation and goals. It is concluded by an overview of the thesis structure.

\subsection{The Higgs Boson and the ATLAS experiments}
2013 the Nobel prize in physics acknowledged the discovery of the Higgs Boson. A particle the physicist Peter Higgs predicted to exist, giving mass to other elementary particles. 

\subsection{The Higgs Boson Machine Learning Challenge}
Kaggle is an internet community of data scientists, it hosts several competitions posed by businesses or organizations. Further services involve open datasets, a "Jobs Board" and "Kaggle Rankings", a scoreboard based on performances of community-members in Kaggles competitions.

\subsubsection{Motivation}


\subsubsection{Goal}

\begin{quote}
The goal of the Higgs Boson Machine Learning Challenge is to explore the potential of advanced machine learning methods to improve the discovery significance of the experiment. No knowledge of particle physics is required. Using simulated data with features characterizing events detected by ATLAS, your task is to classify events into "tau tau decay of a Higgs boson" versus "background."
\cite{higgsChallenge}
\end{quote}

\subsection{Overview}
In Chap. \ref{ch:challenge}, we will describe the structure of the challenges dataset \cite{higgsData} and use simple data-analysis methods, to gain first insight about useful features.
We will understand the evaluation metric AMS and related formulas we can use as objective functions for optimizing our classifiers. 
This will be concluded by deriving the formal problem from the challenges task. 

Chap. \ref{ch:methods}  will use first knowledge about the data to choose simple approaches for classification. After these we will describe several more specific and complex methods.

In Chap. \ref{ch:results} we will use the discussed methods and observe their performance on the challenges data. The thesis closes with a discussion about the approaches and their possible influence on other HEP-applications.
\pagebreak
\clearpage

%%%%%%%%%%%%%%%%%%%%%%%%%%%%%%%%%%%%%%%%%%%%%%%%%%%%%%%%%%%%%%%%%%%%%
% Leerseite bei zweiseitigem Druck
%%%%%%%%%%%%%%%%%%%%%%%%%%%%%%%%%%%%%%%%%%%%%%%%%%%%%%%%%%%%%%%%%%%%%
\ifthenelse{ \( \equal{\zweiseitig}{twoside} \and \not \isodd{\value{page}} \)}
	{\pagebreak \thispagestyle{empty} \cleardoublepage}{\clearpage}