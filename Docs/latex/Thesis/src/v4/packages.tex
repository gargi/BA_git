%%%%%%%%%%%%%%%%%%%%%%%%%%notebook-packages%%%%%%%%%%%%%%%%%%%%%%%%%%%%%%%%%%%%%	
	\usepackage{graphicx} % Used to insert images
    %\usepackage{adjustbox} % Used to constrain images to a maximum size 
    %\usepackage{color} % Allow colors to be defined
    \usepackage[svgnames]{xcolor}
    \usepackage{listings}
    \usepackage{enumerate} % Needed for markdown enumerations to work
    \usepackage{amsmath} % Equations
    \usepackage{amssymb} % Equations
    \usepackage{eurosym} % defines \euro
    \usepackage[mathletters]{ucs} % Extended unicode (utf-8) support
    %\usepackage[utf8x]{inputenc} % Allow utf-8 characters in the tex document
    
%%%%%%%%%FOLLOWING HAS TO BE IN THIS VERY ORDER!!!%%%%%%%%%%
    \usepackage{fancybox,fancyvrb} % verbatim replacement that allows latex

    \usepackage{grffile} % extends the file name processing of package graphics 
                         % to support a larger range 
    % The hyperref package gives us a pdf with properly built
    % internal navigation ('pdf bookmarks' for the table of contents,
    % internal cross-reference links, web links for URLs, etc.)
    \usepackage{hyperref}
    \usepackage{longtable} % longtable support required by pandoc >1.10
    \usepackage{booktabs}  % table support for pandoc > 1.12.2
    
    \usepackage{tikz} % Needed to box output/input
    \usepackage{scrextend} % Used to indent output
    \usepackage{needspace} % Make prompts follow contents
    \usepackage{framed} % Used to draw output that spans multiple pages
%%%%%%%%%%%%%%%%%%%%%%%%%%notebook-packages%%%%%%%%%%%%%%%%%%%%%%%%%%%%%%%%%%%%%	

	\usepackage[inner=4cm,outer=2cm]{geometry} % Used to adjust the document margins
    
    \usepackage{ifthen} % ithenelse-command
    \usepackage{color} % allow color-definitions
    \usepackage{fancyhdr}

	%\usepackage[iso]{umlaute}
	\usepackage[utf8x]{inputenc}
	%\usepackage{makeidx} % um ein Index zu erstellen
	
	
%%%%%%%%%%%%%%%%%%%%%%%%%%%%%%%%%%%%% Font %%%%%%%%%%%%%%%%%%%%%%%%%%%%%%%%%%%%%
	
	\usepackage[nottoc]{tocbibind}
	\usepackage[T1]{fontenc} %fuer richtige Trennung bei Umlauten
	\usepackage{lmodern}            % have lmtt loaded; but...
	\usepackage{palatino} % palatino Schriftart, %%nicht kompatibel mit Jupyter
	%\usepackage{tgpagella}          % use "palatino" as main font, ...
	\renewcommand{\ttdefault}{lmtt} % and use lmtt for teletype family
%%%%%%%%%%%%%%%%%%%%%%%%%%%%%%%%%%%%%%%%%%%%%%%%%%%%%%%%%%%%%%%%%%%%%%%%%%%%%%%%	
		\usepackage{shortvrb}

	\usepackage{a4wide} % ganze A4 Weite verwenden
	
	%\usepackage{babelbib} %BibTex auf englisch
	%\selectbiblanguage{english}


	
	
	%\ifpdf
	%\usepackage[pdftex,xdvi]{graphicx}
	%\usepackage{thumbpdf} %thumbs fuer Pdf
	%\usepackage[pdfstartview=FitV]{hyperref} %anklickbares Inhaltsverzeichnis
	%\else
	%\usepackage[dvips,xdvi]{graphicx}
	%\usepackage{graphicx}
	%\usepackage{hyperref} %anklickbares Inhaltsverzeichnis
	%\fi