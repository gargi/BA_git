The evaluation of a single submission to the challenge is related to the common practice in particle physics to rate a discovery by its statistical significance, in this case 

%hier evtl mehr über die Entstehung?

\begin{equation}\label{eq:Z}
	Z = \sqrt{2 \left(n \ln{\left( \frac{n}{\mu_b} \right)} -
	n + \mu_b \right)}
\end{equation}

where $n$ is the total number of observed events and $\mu_b$ is the expected number of background-events.\\
Often in particle physics a significance of at least Z=5 (a five-sigma effect) is regarded as sufficient to claim a discovery \cite{higgsPaper}.

By estimating $n=s+b$ and $mu_b = b$ in Eq. \eqref{eq:Z}, we get the $Approximate$ $Median$ $Significance$ (AMS)

\begin{equation}\label{eq:AMS_1}
	AMS = \sqrt{2 \left( \left( s+b \right) \ln{ \left(1+ \frac{s}
	{b}  \right)} - s \right)}
\end{equation}

which is used by high-energy physicists for optimizing the selection region for stronger discovery significance \cite{higgsPaper}. 

For the challenge, a regularization-term $b_{reg}$ was introduced as an artificial shift to $b$ to decrease variance of the AMS, as this makes it easier to compare the participants if the optimal signal region was small. "The value $b_{reg}=10$ was determined using preliminary experiments." \cite{higgsPaper}

This addition to Eq. \eqref{eq:AMS_1} makes the final evaluation-formula complete:

\begin{equation}\label{eq:AMS_2}
	AMS_2 = \sqrt{2 \left( \left( s+b+b_{reg} \right) \ln{ \left(1+ \frac{s}
	{b+b_{reg}}  \right)} - s \right)}
\end{equation}

For simplicity, we will call it just AMS, as Eq. \eqref{eq:AMS_1} will not have further appearances in this thesis.

\subsubsection{The Leaderboards}
%testdata split in private and public set => KaggleSet
%weights are normalized => KaggleWeight
%compare public and private LB (graphic analysis on forums)

\subsubsection{Alternative objective functions}
For classification, a data scientist wants to train a classifier on an \textit{objective function}. Properties of the AMS (like using the logarithm) make it difficult to use it as objective function, some alternatives were proposed by the challenge-creators \cite{higgsPaper} and some challenge-participants via the Kaggle-Forum \cite{higgsForum}

\begin{itemize}
	\item alternative objective function: $ \frac{s}{\sqrt{b}} $
	\begin{itemize}
		\item only valid when $s \ll b \ and \ b \gg 1$
	\end{itemize}
\end{itemize}



\begin{equation}\label{eq:Z}
	Z = \sqrt{q_0} = \sqrt{2 \left(n \ln{\left( \frac{n}{\mu_b} \right)} -
	n + \mu_b \right)}
\end{equation}

\begin{equation}\label{eq:AMS_3}
	\frac{s}{\sqrt{b}}
\end{equation}



\eqref{eq:AMS_2}
\eqref{eq:AMS_3}
\eqref{eq:Z}