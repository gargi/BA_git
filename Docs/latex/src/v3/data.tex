All data provided by the challenge- and the opencern-dataset \cite{higgsData} was created by the official ATLAS full detector simulator in a two-part-process. The simulator first reproduces proton-proton collisions. Then it tracks these via a virtual model of the ATLAS-detector, the resulting data emulates the statistical properties of the real events.
Signal-events are generated by Higgs Boson production, background-events originate from three known processes, which produce radiation similar to the signal.
\cite{higgsPaper}

about weights...

PRI and DER


One might expect decent physics-knowledge as key in succeeding in the challenge, the top-participants did not use a lot domain-knowledge for feature- or method-selection. One goal of the challenges organization was to set a task for datascientists without any physics-background.\cite{higgsPaper}

The features $Weight$ and $Label$ were originally only provided in the training-dataset. The data used in this thesis is expanded by complete $Weight$-, $Label$-features and the Kaggle-specific features $KaggleSet$ and $KaggleWeight$.

All features of the dataset are described in Appendix A.

\subsubsection{Simple feature-selection}
scatterplots...
histograms?