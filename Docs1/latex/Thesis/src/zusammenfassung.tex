\begin{titlepage}
\vspace*{\fill}

%%% Die folgende Zeile nicht ändern!
\section*{\ifthenelse{\equal{\sprache}{deutsch}}{Zusammenfassung}{Abstract}}
%%% Zusammenfassung:
This bachelor thesis presents the task of event selection for data analysis performed by the ATLAS experiment at CERN. A good solution of this classification problem is a key to successfully claim a discovery in particle physics, like the Higgs Boson discovery in 2012.\\
The Higgs Boson Machine Learning Challenge, hosted on Kaggle, serves as an example for this task and is further investigated in this work. Several strategies are presented. The Gradient Boosting package XGBoost stood out with good prediction while having excellent runtime. Though it did not provide the winning submission it was acknowledged by the competition hosts with a special jury award.
As event selection is part of a budgeted data analysis system in ATLAS, runtime is an important factor for algorithms to be used in actual CERN applications. It is concluded that XGBoost has a good chance in replacing common tools in particle physics.

While there have been original tools created for testing and analysing several approaches to solve the task, no new classification algorithms have been designed or programmed in this work. The focus is set primarily on the analysis of existing classifiers provided by existing packages, like Logistic Regression Classification of scikit-learn, a machine learning toolbox for Python. 

\vspace{25 mm}

\textbf{Keywords:} Higgs  boson, Kaggle, Machine learning, Classification, Data science, scikit-learn

\vspace*{\fill}
\end{titlepage}
\ifthenelse{\equal{\zweiseitig}{twoside}}{\clearpage\begin{titlepage}
~\end{titlepage}}{}